%\documentclass[10pt,notes]{beamer}       % print frame + notes
%\documentclass[10pt, notes=only]{beamer}   % only notes
\documentclass[11pt]{beamer}              % only frames

%%%%%% IF YOU WOULD LIKE TO CREATE LECTURE NOTES COMMENT OUT THE FOlLOWING TWO LINES
%\usepackage{pgfpages}
%\setbeameroption{show notes on second screen=bottom} % Both

\usepackage{graphicx}
\DeclareGraphicsExtensions{.pdf,.png,.jpg}
\usepackage{color}
\usetheme{winslab}
\usepackage[utf8]{inputenc}
\usepackage[english]{babel}
\usepackage{amsmath}
\usepackage{amsfonts}
\usepackage{amssymb}




\usepackage{algorithm2e,algorithmicx,algpseudocode}
\algnewcommand\Input{\item[\textbf{Input:}]}%
\algnewcommand\Output{\item[\textbf{Output:}]}%
\newcommand\tab[1][1cm]{\hspace*{#1}}

\algnewcommand{\Implement}[2]{\item[\textbf{Implements:}] #1 \textbf{Instance}: #2}%
\algnewcommand{\Use}[2]{\item[\textbf{Uses:}] #1 \textbf{Instance}: #2}%
\algnewcommand{\Trigger}[1]{\Statex{\textbf{Trigger:} (#1)}}%
\algnewcommand{\Events}[1]{\item[\textbf{Events:}] #1}%
\algnewcommand{\Need}[1]{\item[\textbf{Needs:}] #1}%
\algnewcommand{\Event}[2]{\Statex \item[\textbf{On#1:}](#2) \textbf{do}}%
\algnewcommand{\Trig}[3]{\State \textbf{Trigger}  #1.#2 (#3) }%
\def\true{\textbf{T}}
\def\false{\textbf{F}}


\author[Selay Tekgül]{Selayy Tekgül\\\href{mailto:selay@ceng.metu.edu.tr}{selay@ceng.metu.edu.tr}}
%\author[J.\,Doe \& J.\,Doe]
%{%
%  \texorpdfstring{
%    \begin{columns}%[onlytextwidth]
%      \column{.45\linewidth}
%      \centering
%      John Doe\\
%      \href{mailto:john@example.com}{john@example.com}
%      \column{.45\linewidth}
%      \centering
%      Jane Doe\\
%      \href{mailto:jane.doe@example.com}{jane.doe@example.com}
%    \end{columns}
%  }
%  {John Doe \& Jane Doe}
%}

\title[Coordinator Election Algorithms in Ring Topology]{Coordinator Election Algorithms in Ring Topology}
\subtitle[Short SubTitle]{Chang Roberts and Franklin's Algorithms}
%\date{} 

\begin{document}

\begin{frame}[plain]
\titlepage
\note{In this talk, I will present .... Please answer the following questions:
\begin{enumerate}
\item Why are you giving presentation?
\item What is your desired outcome?
\item What does the audience already know  about your topic?
\item What are their interests?
\item What are key points?
\end{enumerate}
}
\end{frame}

\begin{frame}[label=toc]
    \frametitle{Outline of the Presentation}
    \tableofcontents[subsubsectionstyle=hide]
\note{ The possible outline of a talk can be as follows.
\begin{enumerate}
\item Outline 
\item Problem and background
\item Design and methods
\item Major findings
\item Conclusion and recommendations 
\end{enumerate} Please select meaningful section headings that represent the content rather than generic terms such as ``the problem''. Employ top-down structure: from general to more specific.
}
\end{frame}
%
%\part{This the First Part of the Presentation}
%\begin{frame}
%        \partpage
%\end{frame}
%
\section{The Problem}
%\begin{frame}
%        \sectionpage
%\end{frame}

\begin{frame}{Distributed coordinator/leader election}
\framesubtitle{Imagine a network of data storage servers arranged in a closed loop, like a digital \alert{RING}.
Each server in this ring holds a piece of crucial information.

However, a critical task arises: the servers need to agree on a single coordinator to manage updates and ensure data consistency across the entire ring.
Without a designated leader, conflicts could occur. Different servers might update the same information simultaneously, leading to inconsistencies and errors.}
\begin{block}{The Ring Election Problem} 
The Ring Election problem asks: how can these servers elect a single coordinator from within the ring itself?
The chosen coordinator will be responsible for coordinating updates, ensuring all servers possess the same, up-to-date information.
This seemingly simple task requires a \textbf{sophisticated algorithm} to function efficiently within the closed-loop structure of the ring network.\end{block}
\note{}
\end{frame}

\section{The Contribution - Chang Roberts}
\begin{frame}
\frametitle{Chang-Roberts Algorithm: Efficient Leader Election in Directed Rings}
\framesubtitle{}
The Chang-Roberts algorithm offers a \textbf{ROBUST} approach to leader election in directed ring networks, characterized by the following key strengths:
\begin{itemize}
\item Directed Rings: Works best in directed ring networks for optimized communication.
\item ID-based Election: Uses process IDs (highest wins) for efficient leader selection.
\item Guaranteed Termination: Ensures the election concludes with a single leader.
\item Scalable Messages: Message complexity scales proportionally with the network size.
\end{itemize}
\end{frame}


\section{The Contribution - Franklin's}
\begin{frame}
\frametitle{Franklin's Algorithm: Leader Election in Bidirectional Rings}
\framesubtitle{}
Franklin's algorithm tackles leader election in \textbf{BIDIRECTIONAL} ring networks, offering these key advantages:
\begin{itemize}
\item Bidirectional Communication: Leverages communication in both directions within the ring, potentially accelerating leader selection compared to unidirectional algorithms.
\item Probabilistic Selection: Introduces an element of randomness to potentially break ties and avoid deadlocks in scenarios with identical process IDs.
\item Scalable Messages: Message complexity scales proportionally with the network size, similar to Chang-Roberts.
\item Guaranteed Termination: Ensures the election concludes with a single leader, even in the presence of process failures.
\end{itemize}
\end{frame}


\section{Motivation/Importance}
\begin{frame}
\frametitle{Ring Leader Election: Keeping Order Flowing}
\framesubtitle{``Order from Chaos: Electing a Leader in Ring Networks''}
Ring networks struggle without a leader. Tasks clash, data conflicts, and messages meander.
Leader election establishes a coordinator: streamlining tasks, ensuring data consistency, and optimizing communication.

This is crucial for distributed systems like databases and blockchains, ensuring a smooth flow within the ring.\end{frame}

\section{Background/Model/Definitions/Previous Works}


\subsection{Model, Definitions}

\frame{
\frametitle{Model, Definitions}
\framesubtitle{Distributed coordinator/leader election}
Ring Networks - Leader Election
Imagine a circle of computers (processes) talking to their neighbors. This is a ring network.
\begin{itemize}

\item Chang-Roberts works in directed rings (one-way talk) while Franklin's tackles bidirectional rings (two-way talk).

\item Leader election picks a coordinator (leader) for the ring to manage tasks, data, and communication.
\end{itemize}

}

\subsection{Background, Previous Works}
\begin{frame}{Background}
Leader election in ring networks has been extensively studied. Several algorithms exist, each with trade-offs. Here's a comparison of Chang-Roberts and Franklin's:
\begin{itemize}

\item Chang-Roberts (1979) is a simple and efficient algorithm for directed rings, but limited to unidirectional communication.
    
\item Franklin's (1982) builds on Chang-Roberts, introducing bidirectional communication and probabilistic tie-breaking.
        
Both algorithms guarantee termination and message complexity scales with the network size.
\end{itemize}

\end{frame}




\section{Contribution}
\subsection{Implementation}
\begin{frame}{Coordinator Election in Rings}
\framesubtitle{Chang Roberts Algorithm}
Designed for rings where processes have unique identifiers. Messages with process IDs are circulated, and a process remains active if its own ID is larger than the IDs it receives. The process with the largest ID eventually becomes the leader.
%\begin{figure}
%     \centering
%     \includegraphics[scale=0.5]{figures/Chick1.png}
%     \caption{Awesome Image}
%     \label{fig:awesome_image}
% \end{figure}
\note{
}
\end{frame}


\subsection{Algorithm}

\begin{frame}
\frametitle{Coordinator Election Algorithm in a Ring Topology}
\framesubtitle{Chang Roberts}
Chang Roberts on a ring topology is presented in Algorithm~\ref{alg:changroberts}.
\begin{center}
    \begin{algorithm}[H]
        \scriptsize
        \def\algorithmlabel{Chang Roberts}
        \caption{\algorithmlabel\ algorithm}
        \label{alg:changroberts}
        \begin{algorithmic}[1]
            \State Initialize process ID and state
            \Event{MessageFromBottom} { $m$ }
                \\
                \If { $m.id > self.id$ } {Become passive}
                \Else {Continue as active}
                \State Broadcast $m$
            \Event{MessageFromTop} { $m$ }  
                \\
                \If { $m.id > self.id$ } {Become passive}
                \Else {Continue as active}
                \State Broadcast $m$
        \end{algorithmic}
    \end{algorithm}
    \end{center}
    \end{frame}

\begin{frame}
\frametitle{Advantages}
\framesubtitle{Efficiency Improvement}
Notably enhances message complexity compared to basic flooding algorithms, contributing to more streamlined communication in ring networks.
\end{frame}

\begin{frame}
    \frametitle{Advantages}
    \framesubtitle{Straightforward Implementation}
    Relatively straightforward to implement due to its clear-cut logic and reliance on simple message circulation principles.
\end{frame}


\begin{frame}
    \frametitle{Disadvantages}
    \framesubtitle{Message Complexity}
    Despite being an improvement over basic flooding algorithms, it still exhibits a worst-case message complexity of O(\(n^2\)), which can become prohibitive for larger networks.
\end{frame}


\begin{frame}
    \frametitle{Disadvantages}
    \framesubtitle{Dependency on Unique Identifiers}
    The algorithm heavily relies on the existence of unique identifiers for each process, limiting its applicability in scenarios where such identifiers are not readily available.
\end{frame}







\subsection{Implementation}
\begin{frame}{Coordinator Election in Rings}
\framesubtitle{Franklin's Algorithm}
Designed for bidirectional rings. It uses the concept of “waves” of messages, where a process can start a wave if its ID is larger than its neighbors. Waves travel in both directions, and collisions resolve in favor of the larger ID. Eventually, the largest ID prevails.

%\begin{figure}
%     \centering
%     \includegraphics[scale=0.5]{figures/Chick1.png}
%     \caption{Awesome Image}
%     \label{fig:awesome_image}
% \end{figure}
\note{
}
\end{frame}


\subsection{Algorithm}

\begin{frame}
\frametitle{Coordinator Election Algorithm in a Ring Topology}
\framesubtitle{Franklin's}
Franklin's on a ring topology is presented in Algorithm~\ref{alg:franklins}.
\begin{center}
    \begin{algorithm}[H]
        \scriptsize
        \def\algorithmlabel{Franklin's}
        \caption{\algorithmlabel\ algorithm}
        \label{alg:franklins}
        \begin{algorithmic}[1]
            \State Initialize process ID and state
            \Event{MessageFromBottom} { $m$ }
                \\
                \If { $m.id > self.id$ } {Start wave}
                \State Broadcast $m$
            \Event{MessageFromTop} { $m$ }  
                \\
                \If { $m.id > self.id$ } {Start wave}
                \State Broadcast $m$
        \end{algorithmic}
    \end{algorithm}
    \end{center}
    \end{frame}

\begin{frame}
\frametitle{Advantages}
\framesubtitle{Optimized Message Complexity}
Offers a significant improvement in terms of message complexity with a worst-case scenario of O(n log n), particularly beneficial for larger ring networks.
\end{frame}

\begin{frame}
    \frametitle{Advantages}
    \framesubtitle{Scalability}
    Due to its efficient message complexity, it scales well with increasing network size, making it suitable for a wide range of applications.
\end{frame}


\begin{frame}
    \frametitle{Disadvantages}
    \framesubtitle{Complex Implementation}
    Implementation of Franklin’s algorithm can be considerably more complex compared to simpler algorithms like Chang and Roberts, requiring a deeper understanding of bidirectional message propagation and collision resolution.
\end{frame}


\begin{frame}
    \frametitle{Disadvantages}
    \framesubtitle{Communication Channel Requirements}
    Relies on bidirectional communication channels between neighboring processes, which may pose challenges in certain network architectures where such channels are not readily available or feasible.
\end{frame}







\section{Experimental results/Proofs}

\subsection{Main Result 1}
\begin{frame}
\frametitle{Main Result 1}
\framesubtitle{}
Choose \textbf{just the key results}. They should be important, non-trivial, should give the flavour of the rest of the technical details and should be presentable in a relatively short period of time. Use figures instead of tables instead of text.

Better to present 10\% the entire audience gets than 90\% nobody gets
\end{frame}


\subsection{Main Result 2}
\begin{frame}
\frametitle{Main Result 2}
\framesubtitle{Try a subtitle}
\begin{itemize}
\item Make sure your notation is clear and consistent throughout the talk. Prepare a slide that explains the notation in detail, in case that is needed or if somebody asks.
\item Always label all of your axes on graphs; use short but helpful captions on figures and tables. It is also very useful to have an arrow on the side which clearly shows which direction is considered better (e.g., "up is better").
\item If you have experimental results, make sure you clearly present the experimental paradigm you used, and the details of your methods, including the number of trials, the specific analysis tools you applied, significance testing, etc.
\item The talk should contain at least a brief discussion of the limitations and weaknesses of the presented approach or results, in addition to their strengths. This, however, should be done in an objective manner -- don't enthusiastically put down your own work.
\end{itemize}
\end{frame}


\subsection{Main Result 3}
\begin{frame}
\frametitle{Main Result 3}
\framesubtitle{}
\begin{itemize}
\item If time allows, the results should be compared to the most related work in the field. You should at least prepare one slide with a summary of the related work, even if you do not get a chance to discuss it. This will be helpful if someone asks about it, and will demonstrate your mastery of the material.
\item Spell check again.
\item Give for each of the x-axis, y-axis, and z-axis
\item Label, unit, scale (if log scale)
\item Give the legend
\item Explain all symbols
\item Take an example to illustrate a specific point in the figure
\end{itemize}
\end{frame}



\section{Conclusions}
\begin{frame}
\frametitle{Conclusions}
\framesubtitle{Hindsight is Clearer than Foresight}
Advices come from \cite{spillman2000present}.
\begin{itemize}
\item You can now make observations that would have been confusing if they were introduced earlier. Use this opportunity to refer to statements that you have made in the previous three sections and weave them into a coherent synopsis. You will regain the attention of the non- experts, who probably didn’t follow all of the Technicalities section. Leave them feeling that they have learned something nonetheless.
\item Give Open Problems It is traditional to end with a list of open problems that arise from your paper. Mention weaknesses of your paper, possible generalizations, and indications of whether they will be fruitful or not. This way you may defuse antagonistic questions during question time.
\item Indicate that your Talk is Over
An acceptable way to do this is to say “Thank-you. Are there any questions?”\cite{einstein}
\end{itemize}

\end{frame}

\section*{References}
\begin{frame}{References}
\tiny
\bibliographystyle{IEEEtran}
\bibliography{refs}
\end{frame}

\begin{frame}{How to prepare the talk?}
Please read \url{http://larc.unt.edu/ian/pubs/speaker.pdf}
\begin{itemize}
\item The Introduction:  Define the Problem,    Motivate the Audience,    Introduce Terminology,    Discuss Earlier Work,    Emphasize the Contributions of your Paper,    Provide a Road-map.
\item The Body:    Abstract the Major Results, Explain the Significance of the Results, Sketch a Proof of the Crucial Results
\item Technicalities: Present a Key Lemma, Present it Carefully
\item The Conclusion: Hindsight is Clearer than Foresight, Give Open Problems, Indicate that your Talk is Over
\end{itemize}

\note{
\begin{itemize}
\item The Introduction:  Define the Problem,    Motivate the Audience,    Introduce Terminology,    Discuss Earlier Work,    Emphasize the Contributions of your Paper,    Provide a Road-map.
\item The Body:    Abstract the Major Results, Explain the Significance of the Results, Sketch a Proof of the Crucial Results
\item Technicalities: Present a Key Lemma, Present it Carefully
\item The Conclusion: Hindsight is Clearer than Foresight, Give Open Problems, Indicate that your Talk is Over 
\end{itemize}
}
\end{frame}



\thankslide




\end{document}